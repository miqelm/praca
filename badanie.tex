\chapter{Badanie warunków środowiskowych i meteorologicznych}
Zbieranie informacji na temat aktualnych danych pogodowych zawsze było przydatne dla ludzkości, a nawet stało się bardzo ważne w dzisiejszych czasach. Różne przedsiębiorstwa z dziedziny automatyki, budownictwa, przemysłu itp. potrzebują tych danych, aby wiedzieć jak zorganizować swoją pracę oraz przewidzieć plan na najbliższą przyszłość. Posiadając aktualne pomiary pogodowe można zmienić plan pracy, jeżeli warunki atmosferyczne na to nie pozwalają.

Przykładowo przeprowadzenie naprawy sieci elektrycznej przy bardzo dużej wilgotności powietrza jest stosunkowo niebezpieczne, ponieważ mogą występować przepięcia, prowadzące do całkowitego jej uszkodzenia. Badanie temperatury otoczenia jest niezmiernie potrzebne przy instalacjach, które wymagają określonego przedziału ciepła, stosuje się to np. w chłodniach. Dzięki pomiarom temperatury można sterować regulatorem ogrzewania lub chłodzenia w celu uzyskania optymalnych i rekomendowanych warunków pracy urządzeń przemysłowych.

Na podstawie aktualnych danych pogodowych wraz z historią pomiarów, dzięki zastosowaniu specjalnych algorytmów, można uzyskać z pewnym prawdopodobieństwem prognozę pogody na najbliższą przyszłość. Wyniki przewidywania pogody nigdy nie są idealne, ale pozwalają na przygotowanie się na możliwe warunki atmosferyczne.

Pomiary warunków środowiskowych są nieodłącznym elementem każdego regulatora automatyki, zastosowanego w klimatyzatorach, lodówkach, grzejnikach, nawilżaczach lub osuszaczach powietrza. Na ich podstawie wyliczane jest sterowanie, które następnie jest podawane na element wykonawczy, np. część chłodzącą. Pomiar aktualnej wartości z wartością zadaną jest porównywany nieustannie w cyklu działania urządzenia.

Projekt inteligentnego domu, w którym niemal wszystko jest regulowane automatycznie, wymaga informacji o aktualnych warunkach i na tej podstawie ustawia poziom ogrzewania, nawilżania powietrza.
