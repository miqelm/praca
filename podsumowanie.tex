\chapter{Podsumowanie}
Projekt inżynierskii opisany w niniejszej pracy ma na celu badanie warunków meteorologicznych badając trzy pomiary: temperaturę, ciśnienie oraz wilgotność powietrza.

Testy przeprowadzone po ukończeniu tworzenia układu oraz oprogramowania dowodzą, że projekt inżynierski został zakończony sukcesem, wszystkie zaplanowane funkcjonalności zostały w 100 \% zrealizowane. Pomiary warunków meteorologiczne, które zostały zmierzone były bardzo zbliżone do wskazań otrzymanych przy pomocy standardowych mierników, jak np. termometr pokojowy.

Największym problemem napotkamym podczas tworzenia niniejszego projektu było doprowadzenie do kompilacji pełnego projektu, wraz ze wszystkimi bibliotekami, cross-kompilowaniem oraz uruchamianiem aplikacji na innej architekturze niż na tej, na której tworzone było oprogramowanie.

Możliwymi koncepcjami rozwinięcia tematyki niniejszej pracy dyplomowej jest stworzenie i uruchomienie algorytmów do przewidywania pogody. Jest to zagadanienie bardzo trudnie i~skomplikowanie, jednakże na pewno bardzo interesujące oraz pochłaniające. Numeryczne obliczanie prognozy pogody może bardzo ułatwić użytkownikowi życie.

Inną możliwością jest zastosowanie mikrokomputera BeagleBone oraz czujnika ciśnienia atmosferycznego do zbudowania urządzenia latającego, np. kwadrokoptera.