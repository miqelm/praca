\chapter{Przechowywanie oraz wyświetlanie wyników}

Pobrane dane z czujników muszą być przechowane, aby mogły być później odczytane i przeanalizowane. Jest kilka sposobów na gromadzenie danych, najczęściej używanymi jest korzystanie z plików tekstowych lub baz danych. W projekcie inżynierskim wykorzystana została baza danych MySQL. 

\section{Baza danych MySQL}
MySQL jest systemem bazy danych opartych na strukturalnym języku zapytań SQL. Używa on tego języka do zapisywania oraz pobierania informacji, zarówno z, jak i do bazy. 
Pomiary ze stworzonej stacji pogodowej zostają wysyłane nieustannie, w trakcie działania programu, do bazy danych MySQL. Znajduje się ona na serwerze AGH i została skonfigurowana poprzez stronę http://mysql.agh.edu.pl. Baza przechowuje wartości wszystkich zmierzonych parametrów meteorologicznych od pierwszego uruchomienia stacji pogody.

Struktura bazy danych jest prosta, posiada ona tabelę o nazwie "weather\_station". To właśnie w niej bezpośrednio przechowywane są wartości pomiarowe.

Składa się ona z sześciu kolumn:
\begin{enumerate}
\item godzina wykonania pomiaru i wysłania do bazy
\item data
\item temperatura z czujnika DHT-22
\item wilgotność z czujnika DHT-22
\item temperatura z czujnika BMP085
\item ciśnienie z czujnika BMP085
\end{enumerate}

\section{Strona internetowa z wynikami pomiarów}

\section{Interfejs użytkownika}
Strona internetowa, która została stworzona specjalnie do wizualizacji wyników ma bardzo prosty i intuicyjny interfejs użytkownika.

Główną jego częścią są ostatnie wskazania z czujników. Poniżej tych znajduje się wykres historii pomiarów każdego parametru. Wykres jest dowolnie konfigurowalny, użytkownik może wybrać, które wartości mają zostać pokazane, a które mają być ukryte. Dodatkowym udogodnieniem jest możliwość regulowania zakresem daty, w którym mają zostać pokazane pomiary.
