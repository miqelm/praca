\chapter{Magistrale szeregowe}

\section{Magistrala $\mathbf{I^{2}C}$}
Nazwa jest to akronimem od Inter-Intergrated Circuit. Standard został opracowany w latach osiemdziesiątych przez firmę Philips.

Jest ona bardzo często wykorzystywana w układach mikroprocesorowych, w sterownikach wyświetlaczy LCD, można ją stosować do sterowania pamięci RAM, EPROM, układami I/O.

Zaletami magistrali $I^{2}C$ są niewątpliwie takie właściwośći jak: odporność na zakłócenia zewnętrzne, dodtakowe układu podłączone do niej mogą być dodawane lub wyłączone bez ingerencji w pozostały układ połączeń wcześniej stworzonych, połączenie na magistrali składają się tylko z dwóch przewodów, przez co ich ogólna liczba jest minimalizowana, wykrywanie błędów jest proste i łatwe do analizy, na magistrali może znajdować się wiele urządzeń typu master, umożliwiając kontrolę gotowych układów przez zewnętrzny komputer.

Magistrala $I^{2}C$ posiada dwie dwukierunkowe linie: dane są przesyłane przez Serial Data (SDA), natomiast sygnał zegara na Serial Clock (SCL). 

\section{Magistrala One-Wire}