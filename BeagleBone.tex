\chapter{Mikrokomputer BeagleBone Black}
Projekt inżynierski został zrealizowany, w głównej części, na mikrokomputerze BeagleBone Black. Został on stworzony specjalnie z myślą o programistach OpenSource oraz tych, dla których liczy się niskie zużycie energii. Jest to oparta na procesorze AM335x ARM Cortex-A8, taktowany częstotliwością 1 GHz, płytka developerska, która została wyposażona w 512 MB pamięci RAM, 2 GB pamięci FLASH, akcelerator grafiki 3D. Posiada szereg różnych interfejsów, takich jak: HDMI, USB, Ethernet, czytnik kart microSD. BeagleBone można zasilać na dwa sposoby, pierwszy - poprzez kabel USB podłączony do USB (5V) albo przy użyciu zewnętrznego zasilacza, również 5V. Dla użytkownika zostały również wyprowadzone 96 pinów typu wejśćie/wyjście.

Na mikrokomputerze można zainstalować i ze swobodą korzystać z najpopolarniejszych dystrybucji Linuxa, np. Ubuntu, Debian, Fedora, Arch. Istnieje również możliwość uruchomienia na BeagleBone systemu Android.

