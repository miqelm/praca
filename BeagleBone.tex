\chapter{Mikrokomputer BeagleBone Black}
Praca została zrealizowana, w głównej części, na mikrokomputerze BeagleBone Black. Został on zbudowany specjalnie z myślą o programistach OpenSource oraz tych, dla których liczy się niskie zużycie energii. Jest to oparta na procesorze AM335x ARM Cortex-A8, taktowanym częstotliwością 1 GHz, płyta developerska, która została wyposażona w 512 MB pamięci RAM, 2 GB pamięci FLASH, akcelerator grafiki 3D. Posiada szereg różnych interfejsów, takich jak: HDMI, USB, Ethernet, czytnik kart microSD. BeagleBone'a można zasilać na dwa sposoby, pierwszy - poprzez kabel USB podłączony do USB (5V) albo przy użyciu zewnętrznego zasilacza, również 5V. Dla użytkownika zostało również wyprowadzone 96 pinów typu wejście/wyjście.

Na mikrokomputerze można zainstalować i ze swobodą korzystać z najpopolarniejszych dystrybucji Linuxa, np. Ubuntu, Debian, Fedora, Arch. Istnieje również możliwość uruchomienia na nim systemu Android.

\begin{figure}[h]
\centering
\includegraphics[scale=0.5]{beaglebone}
\caption{BeagleBone Black}
\label{fig:beaglebone}
\end{figure}

Po zakupie, domyślnie BeagleBone posiada zainstalowaną dystrybucją Linuxa - Ångström. Z uwagi na większą znajomość innego systemu operacyjnego - Ubuntu, na mikrokomputerze została zainstalowana, na potrzeby pracy, właśnie ta dystrybucja w wersji 13.04, dedykowana na platformę ARM Hard Float. Platforma ta posiada zaimplementowaną sprzętową obsługę liczb zmiennoprzecinkowych. Obraz systemu operacyjnego oraz instrukcję jego zainstalowania za pomocą karty pamięci microSD można odnaleźć na głównej stronie projektu ARM HF: www.armhf.com.

BeagleBone został podłączony do komputera z zainstalowanym środowiskiem programistycznym przy użyciu portu USB. Po zainstalowaniu odpowiednich sterowników, które znajdują się na oficjalnej stronie producenta tej płyty oraz pamięci FLASH, port ten jest wykrywany jako interfejs sieciowy i tworzona jest sieć łącząca komputer z urządzeniem. Domyślne ustawienia sprawiają, że do BeagleBone'a można podłączyć się przy użyciu protokołu SSH, łącząc się z adresem: 192.168.7.2. Po poprawnym zalogowaniu się do płyty poprzez program \emph{ssh}, dostępny na Ubuntu, zostanie wyświetlony ekran podobny do tego poniżej:

\begin{figure}[h]
\centering
\includegraphics[scale=0.5]{konsola}
\caption{Zrzut ekranu z konsoli}
\label{fig:konsola}
\end{figure}